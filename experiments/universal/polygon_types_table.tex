\documentclass[11pt,a4paper]{article}

\usepackage[utf8]{inputenc}
\usepackage{amsmath,amssymb}
\usepackage{geometry}
\usepackage{booktabs}
\usepackage{longtable}
\usepackage{hyperref}
\usepackage{array}

\geometry{margin=0.75in}

\title{Catalog of Polygon Classes with O(n) Triangulation Algorithms}
\author{Universal Triangulation Project}
\date{\today}

\begin{document}

\maketitle

\section{Complete Taxonomy of Polygon Classes}

The following table catalogs all known polygon classes that admit $O(n)$ triangulation algorithms. These classes together span the entire space of simple polygons.

\begin{center}
\renewcommand{\arraystretch}{1.4}
\begin{longtable}{>{\raggedright}p{2.5cm} p{4.5cm} p{3cm} p{1.5cm} p{2.5cm}}
\toprule
\textbf{Polygon Class} & \textbf{Definition} & \textbf{Algorithm} & \textbf{Year} & \textbf{Reference} \\
\midrule
\endfirsthead

\toprule
\textbf{Polygon Class} & \textbf{Definition} & \textbf{Algorithm} & \textbf{Year} & \textbf{Reference} \\
\midrule
\endhead

\midrule
\multicolumn{5}{r}{\textit{Continued on next page...}} \\
\endfoot

\bottomrule
\endlastfoot

\textbf{Convex} & All interior angles $\leq 180^\circ$; all diagonals internal & Fan triangulation from any vertex & Ancient & Folklore \\

\textbf{Star-Shaped} & Exists kernel point $k$ visible from all boundary points & Connect kernel to all vertices & 1975 & Shamos \\

\textbf{Y-Monotone} & Any horizontal line intersects polygon in at most one segment & Stack-based linear sweep & 1978 & Garey et al. \\

\textbf{X-Monotone} & Any vertical line intersects polygon in at most one segment & Stack-based linear sweep & 1978 & Garey et al. \\

\textbf{Monotone (general)} & Monotone w.r.t.\ some direction $d$ & Rotate to axis-aligned, apply y-monotone & 1978 & Garey et al. \\

\textbf{Orthogonal} & All edges axis-aligned (horizontal/vertical) & Horizontal decomposition into monotone & 1982 & Chazelle \\

\textbf{Spiral (Funnel)} & Single reflex chain between two convex endpoints & Linear visibility triangulation & 1982 & Chazelle \\

\textbf{Edge-Visible} & Every interior point visible from some point on edge $e$ & Visibility-based triangulation & 1981 & ElGindy \& Avis \\

\textbf{Weakly Edge-Visible} & Every boundary point visible from some point on edge $e$ & Weak visibility decomposition & 1983 & Avis \& Toussaint \\

\textbf{L-Shaped} & Orthogonal polygon with at most one reflex vertex & Direct decomposition & 1980 & Various \\

\textbf{Staircase} & Orthogonal polygon with reflex vertices on two opposite corners & Diagonal sweep & 1984 & Lingas \\

\textbf{Histogram} & Y-monotone orthogonal polygon (rectilinear monotone) & Specialized monotone algorithm & 1984 & Fournier \& Montuno \\

\textbf{Unimodal} & Boundary can be split into two chains: one increasing, one decreasing & Equivalent to monotone & 1978 & Garey et al. \\

\textbf{Simple (General)} & Non-self-intersecting boundary & Trapezoidalization with polygon cutting & 1991 & Chazelle \\

\end{longtable}
\end{center}

\section{Class Inclusion Relationships}

The classes form a hierarchy with the following inclusions:

\begin{center}
\begin{tabular}{ll}
\toprule
\textbf{Inclusion} & \textbf{Relationship} \\
\midrule
Convex $\subset$ Star-Shaped & Every convex polygon is star-shaped (kernel = interior) \\
Convex $\subset$ Monotone & Every convex polygon is monotone in all directions \\
Star-Shaped $\subset$ Simple & Star-shaped is a special case of simple \\
Monotone $\subset$ Simple & Monotone is a special case of simple \\
Spiral $\subset$ Edge-Visible & Spiral polygons are visible from their base edge \\
Edge-Visible $\subset$ Simple & Edge-visible is a special case of simple \\
Orthogonal $\cap$ Monotone $\neq \emptyset$ & Histograms are both orthogonal and monotone \\
L-Shaped $\subset$ Orthogonal & L-shaped is a restricted orthogonal \\
Staircase $\subset$ Orthogonal & Staircase is a restricted orthogonal \\
Histogram $\subset$ Orthogonal $\cap$ Monotone & Histograms are monotone orthogonal \\
\bottomrule
\end{tabular}
\end{center}

\section{Detection Algorithms}

Each polygon class can be detected in $O(n)$ time:

\begin{center}
\renewcommand{\arraystretch}{1.3}
\begin{tabular}{lll}
\toprule
\textbf{Class} & \textbf{Detection Method} & \textbf{Complexity} \\
\midrule
Convex & Check all cross products have same sign & $O(n)$ \\
Star-Shaped & Compute kernel via half-plane intersection & $O(n)$ \\
Y-Monotone & Check both chains strictly decrease/increase in $y$ & $O(n)$ \\
X-Monotone & Check both chains strictly decrease/increase in $x$ & $O(n)$ \\
Orthogonal & Check all edges are horizontal or vertical & $O(n)$ \\
Spiral & Find leftmost/rightmost, check single reflex chain & $O(n)$ \\
Edge-Visible & Compute weak visibility polygon from edge & $O(n)$ \\
Histogram & Check orthogonal + y-monotone & $O(n)$ \\
Simple & Shamos-Hoey intersection detection & $O(n \log n)^*$ \\
\bottomrule
\end{tabular}
\end{center}

$^*$Simple polygon detection requires $O(n \log n)$, but if input is guaranteed simple, this step is unnecessary.

\section{Algorithm Details by Class}

\subsection{Fan Triangulation (Convex)}

\begin{verbatim}
FanTriangulate(P):
    v0 = P[0]
    for i = 2 to n-1:
        output triangle(v0, P[i-1], P[i])
\end{verbatim}

\textbf{Triangles produced}: $n-2$

\textbf{Complexity}: $O(n)$

\subsection{Kernel Triangulation (Star-Shaped)}

\begin{verbatim}
KernelTriangulate(P):
    K = ComputeKernel(P)  // O(n) half-plane intersection
    k = any point in K
    for i = 0 to n-1:
        output triangle(k, P[i], P[(i+1) mod n])
\end{verbatim}

\textbf{Note}: Produces $n$ triangles (including one vertex at kernel), which can be reduced to $n-2$ by merging.

\textbf{Complexity}: $O(n)$

\subsection{Stack-Based Triangulation (Monotone)}

\begin{verbatim}
MonotoneTriangulate(P):
    Sort vertices by y-coordinate (already sorted for monotone)
    Initialize stack S with first two vertices
    for each remaining vertex v:
        if v on opposite chain from stack top:
            pop all from S, add diagonals to v
            push previous vertex and v
        else:
            while S.size >= 2 and diagonal(v, S.second) is valid:
                pop S, add diagonal
            push v
    Add diagonals from last vertex to remaining stack elements
\end{verbatim}

\textbf{Complexity}: $O(n)$ (each vertex pushed/popped once)

\subsection{Trapezoidalization (General Simple)}

Chazelle's algorithm:
\begin{enumerate}
    \item Build a \emph{polygon-cutting} hierarchy
    \item Compute horizontal trapezoids in $O(n)$ without sorting
    \item Each trapezoid is a ``mountain'' (monotone with one edge as base)
    \item Triangulate each mountain in $O(1)$ amortized per triangle
\end{enumerate}

\textbf{Complexity}: $O(n)$ deterministic

\textbf{Implementation difficulty}: Very high (53-page paper)

\section{Practical Recommendations}

\begin{center}
\begin{tabular}{lp{8cm}}
\toprule
\textbf{Input Type} & \textbf{Recommended Algorithm} \\
\midrule
Known convex & Fan triangulation ($O(n)$, trivial) \\
Known monotone & Stack-based ($O(n)$, simple) \\
Unknown structure, $n < 1000$ & Ear clipping ($O(n^2)$ but simple and fast) \\
Unknown structure, $n \geq 1000$ & Seidel randomized ($O(n \log^* n)$ expected) \\
Worst-case guarantee needed & Chazelle ($O(n)$ but complex) \\
Near-convex (few reflex) & Reflex-aware ear clipping (this project) \\
\bottomrule
\end{tabular}
\end{center}

\section{Historical Timeline}

\begin{center}
\begin{tabular}{lll}
\toprule
\textbf{Year} & \textbf{Result} & \textbf{Authors} \\
\midrule
1975 & Kernel computation for star-shaped & Shamos \\
1978 & $O(n)$ for monotone polygons & Garey, Johnson, Preparata, Tarjan \\
1978 & $O(n \log n)$ decomposition into monotone & Garey et al. \\
1981 & $O(n)$ for edge-visible polygons & ElGindy \& Avis \\
1982 & $O(n)$ for orthogonal polygons & Chazelle \\
1982 & Spiral (funnel) triangulation & Chazelle \\
1984 & Ear clipping formalized & Meisters \\
1986 & $O(n \log \log n)$ triangulation & Tarjan \& Van Wyk \\
1988 & $O(n \log^* n)$ randomized & Clarkson, Tarjan, Van Wyk \\
1991 & $O(n \log^* n)$ randomized (simplified) & Seidel \\
1991 & $O(n)$ deterministic (general simple) & Chazelle \\
\bottomrule
\end{tabular}
\end{center}

\section{Open Question (Resolved)}

The question ``Can all simple polygons be triangulated in $O(n)$ time?'' was open from 1975 to 1991.

\textbf{Resolution}: Chazelle (1991) proved $O(n)$ is achievable for all simple polygons.

\textbf{Current status}: The problem is \emph{solved}. However, Chazelle's algorithm remains impractical, and finding a \emph{simple} $O(n)$ algorithm is still open.

\end{document}

