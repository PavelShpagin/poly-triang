\documentclass[11pt]{article}

\usepackage[utf8]{inputenc}
\usepackage{amsmath,amssymb,amsthm}
\usepackage{algorithm}
\usepackage{algpseudocode}
\usepackage[margin=1in]{geometry}
\usepackage{graphicx}

\newtheorem{theorem}{Theorem}
\newtheorem{lemma}[theorem]{Lemma}
\newtheorem{corollary}[theorem]{Corollary}

\title{Randomized Linear-Time Polygon Triangulation\\(Adapting Amato-Goodrich-Ramos)}
\author{}
\date{}

\begin{document}
\maketitle

\begin{abstract}
We present a randomized algorithm for triangulating a simple polygon with $n$ vertices. The algorithm runs in $O(n)$ expected time. It is based on the "Divide-and-Conquer" approach of Amato, Goodrich, and Ramos \cite{amato2000}, which simplifies the sampling strategy of Seidel's algorithm \cite{seidel1991} and avoids the complexity of Chazelle's deterministic method \cite{chazelle1991}. By sampling a small subset of vertices to form a coarse decomposition (trapezoidal map) and "tracing" the polygon boundary through it, we reduce the problem to independent sub-problems of small total expected size.
\end{abstract}

\section{Introduction}

The history of polygon triangulation is marked by a trade-off between theoretical optimality and practical simplicity.
\begin{itemize}
    \item \textbf{Chazelle (1991)}: $O(n)$ deterministic. Extremely complex.
    \item \textbf{Seidel (1991)}: $O(n \log^* n)$ expected. Randomized Incremental. Moderate complexity.
    \item \textbf{Amato et al. (2000)}: $O(n)$ expected. Randomized Divide \& Conquer. Simpler than Chazelle.
\end{itemize}

We describe the Amato-Goodrich-Ramos (AGR) approach as the "Holy Grail" of practical linear-time triangulation.

\section{The Algorithm}

The core insight is to replace the incremental updates of Seidel (which require a DAG or complex history) with a batch Divide-and-Conquer strategy.

\begin{algorithm}
\caption{AGR Triangulation}
\begin{algorithmic}[1]
\State \textbf{Input:} Simple polygon $P$ with $n$ edges.
\State \textbf{Base Case:} If $n < C$, triangulate using Ear Clipping ($O(1)$).
\State \textbf{Step 1: Sample}
\State Pick a random sample $S \subset P$ of size $s = n^\epsilon$ (e.g., $\sqrt{n}$).
\State \textbf{Step 2: Coarse Decomposition}
\State Build the Trapezoidal Map $T(S)$ of the sample.
\State Time: $O(s \log s) \ll O(n)$.
\State \textbf{Step 3: Trace}
\State Trace the boundary of $P$ through $T(S)$.
\State Insert the edges of $P$ into the trapezoids of $T(S)$.
\State Since $P$ is a connected chain, we can walk from one trapezoid to the next in $O(1)$ amortized time (plus intersection costs).
\State \textbf{Step 4: Recurse}
\State The tracing step partitions $P$ into sub-polygons $P_1, \dots, P_k$ inside the trapezoids.
\State Recursively triangulate each $P_i$.
\end{algorithmic}
\end{algorithm}

\section{Theoretical Analysis}

\begin{theorem}
The expected running time of the algorithm is $O(n)$.
\end{theorem}

\begin{proof}[Proof Sketch]
Let $T(n)$ be the expected running time.
\begin{enumerate}
    \item \textbf{Sampling \& Map Construction}: Takes $O(s \log s)$. With $s = n^{1/2}$, this is $O(n)$.
    \item \textbf{Tracing}: The time to trace $P$ through $T(S)$ is proportional to $n$ plus the number of intersections between $P$ and the vertical extensions of $S$. By standard $\epsilon$-net theory (or random sampling properties), the expected number of intersections is $O(n)$.
    \item \textbf{Recursion}: The sum of the sizes of the sub-polygons is $\sum |P_i| = n + O(n) = O(n)$. The maximum size of a sub-problem is expected to be $O(n/s \cdot \log n)$.
\end{enumerate}
The recurrence relation is roughly $T(n) = O(n) + \sum T(n_i)$. Since $\sum n_i = O(n)$ and the problem size decays rapidly, the solution is $T(n) = O(n)$.
\end{proof}

\section{Comparison to Splay Sweep}

The Randomized Splay Sweep we investigated earlier is $O(n)$ for "most" inputs but theoretically $O(n \log n)$ in the worst case (though rotation mitigates this).
The AGR algorithm is **theoretically $O(n)$ expected** for *all* inputs, making it the rigorous choice.

\begin{thebibliography}{9}
\bibitem{amato2000} N. M. Amato, M. T. Goodrich, E. A. Ramos, "Linear-time triangulation of a simple polygon made easier via randomization", SoCG 2000.
\bibitem{seidel1991} R. Seidel, "A simple and fast incremental randomized algorithm...", CGTA 1991.
\end{thebibliography}

\end{document}

