\documentclass[11pt,a4paper]{article}

\usepackage[utf8]{inputenc}
\usepackage[T1]{fontenc}
\usepackage{amsmath,amssymb,amsthm}
\usepackage{algorithm}
\usepackage{algpseudocode}
\usepackage{booktabs}
\usepackage{geometry}

\geometry{margin=1in}

\title{Persistent Chain Method for Polygon Triangulation\\in $O(n + r\log r)$ Deterministic Time}
\author{Research Report}
\date{\today}

\newtheorem{theorem}{Theorem}
\newtheorem{lemma}{Lemma}
\newtheorem{corollary}{Corollary}
\newtheorem{definition}{Definition}
\newtheorem{proposition}{Proposition}

\begin{document}

\maketitle

\begin{abstract}
We present the \emph{Persistent Chain Method} for triangulating simple polygons in $O(n + r\log r)$ deterministic time. The key is maintaining monotonically advancing chain pointers that traverse each edge at most once. By processing reflex vertices in $y$-sorted order and surgically removing processed regions, we achieve provably linear-time chord discovery.
\end{abstract}

\section{Problem}

Let $P$ be a simple polygon with $n$ vertices. A vertex is \emph{upward-reflex} if it is reflex and both neighbors are above it. It is \emph{downward-reflex} if both neighbors are below. Let $r = |R_\uparrow| + |R_\downarrow|$.

\section{Why Naive is $O(n^2)$}

\begin{proposition}
Independent walks from each reflex vertex give $\Omega(n^2)$ worst-case.
\end{proposition}

\begin{proof}
With $r$ nested valleys, the outer valley walk traverses $\Omega(n)$ edges. Total: $\Omega(nr)$.
\end{proof}

\section{The Method}

\subsection{Key Insight}

After processing reflex vertex $v$, we \textbf{remove} the cut-off region (valley or peak) from the polygon. Removed edges cannot be traversed again.

\subsection{Algorithm}

\begin{enumerate}
    \item Build polygon as doubly-linked list: $O(n)$
    \item Identify and sort reflex vertices: $O(n + r\log r)$
    \item Process upward vertices bottom-to-top:
    \begin{itemize}
        \item Walk to find support edges
        \item Splice out valley, add to regions list
    \end{itemize}
    \item Process downward vertices top-to-bottom (symmetric)
    \item Triangulate all $y$-monotone regions: $O(n)$
\end{enumerate}

\subsection{Splicing}

When we find left support edge $e_L = (a_L, b_L)$ and right support $e_R = (a_R, b_R)$:
\begin{itemize}
    \item Create chord endpoints $p_L, p_R$ on these edges
    \item \textbf{Valley}: cycle through $p_L \to a_L \to \cdots \to v \to \cdots \to a_R \to p_R \to p_L$
    \item \textbf{Remaining}: $\cdots \to b_L \to p_L' \to p_R' \to b_R \to \cdots$
\end{itemize}

The valley is removed from the main polygon. Its edges are never traversed again.

\section{Complexity}

\begin{theorem}
Total edge traversals across all walks is $O(n)$.
\end{theorem}

\begin{proof}
Use potential $\Phi = $ edges in remaining polygon. Initially $\Phi_0 = n$.

When processing $v$, let $k$ = edges traversed. After splicing, these $k$ edges are removed (in the valley). 

Amortized cost $= k + \Delta\Phi = k + (-k + O(1)) = O(1)$.

Total: $O(r)$ amortized $+ \Phi_0 = O(n)$ actual.
\end{proof}

\begin{corollary}
Total time is $O(n + r\log r)$.
\end{corollary}

\section{Correctness}

\begin{enumerate}
    \item Chords are valid (inside polygon)
    \item Chords don't intersect (parallel at different heights)
    \item Regions are $y$-monotone (reflex vertices removed)
    \item Monotone triangulation is correct
\end{enumerate}

\section{Critical Lemma}

\begin{lemma}
All edges traversed during walk are in the cut-off region.
\end{lemma}

\begin{proof}
Walking from $v.\text{prev}$ (above $v$) backward, we traverse edges until finding one crossing $y(v)$. The traversed edges connect vertices that eventually dip below $y(v)$. After splicing, these vertices (and edges) are in the valley.
\end{proof}

\section{Conclusion}

The Persistent Chain Method achieves $O(n + r\log r)$ by:
\begin{enumerate}
    \item Processing in sorted $y$-order
    \item Surgically removing processed regions
    \item Guaranteeing each edge traversed at most once
\end{enumerate}

\end{document}

