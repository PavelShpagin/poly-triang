\documentclass[11pt]{article}

\usepackage[margin=1in]{geometry}
\usepackage[hidelinks]{hyperref}

\setlength{\parindent}{0pt}
\setlength{\parskip}{0.7em}

\begin{document}

\begin{flushright}
January 26, 2026
\end{flushright}

Dear Editors,

Please consider our manuscript, \emph{``A Simple, Practical, and Fast Polygon Triangulation in \(O(n + k\log k)\) Time,''} for publication in \emph{Computational Geometry: Theory and Applications (CGTA)}.

This submission is a substantially revised version of our previous manuscript (titled \emph{``Practical polygonal triangulation in \(O(n + r\log r)\) time''}). We rewrote the paper for clarity and strengthened both the theory and the experimental evaluation.

\textbf{Summary of contributions.}
We present a simple, implementation-friendly, deterministic triangulation algorithm for simple polygons with \(n\) vertices running in \(O(n + k\log k)\) time, where \(k\) is the number of local maxima (equivalently local minima) with respect to the sweep direction.
In our benchmarks, it is the fastest among the compared practical implementations.
The algorithm is designed to be practical: the sweep performs balanced-tree work only at the \(2k\) extremal events, while all regular vertices are handled implicitly by amortized pointer advancement along monotone chains.

\textbf{Improvements over the previous submission.}
\begin{itemize}
  \item \textbf{Simpler and more faithful presentation.} The decomposition is presented directly in terms of split/merge event handling (with a chain-centric status structure). The example figure and pseudocode are aligned with this formulation, and the correctness argument is streamlined by proving equivalence to the classical monotone-decomposition sweep.
  \item \textbf{Stronger, cleaner theorems.} We analyze the running time in terms of the event complexity \(k\), and prove \(k \le r+1\) (so the new bound strictly subsumes the previous \(O(n + r\log r)\) statement). We also bound the number of inserted diagonals by the number of split/merge vertices.
  \item \textbf{Implementation and reproducibility.} We provide a complete C++ implementation and a reproducibility package (build scripts, benchmark harness, and sanity checks such as non-crossing diagonals and outputting exactly \(n-2\) triangles).
  \item \textbf{Clear benchmark comparisons and speedups.} We compare against widely used practical baselines with public implementations and report mean \(\pm\) standard deviation timings. The updated results show clear wall-clock speedups (e.g., roughly \(20\times\) over Garey et al.\ on convex inputs at \(n=10{,}000\), and about \(2\times\) on low-\(k\) ``dent'' polygons; on random inputs our method is fastest across the reported sizes).
\end{itemize}

Thank you for your time and consideration.

Sincerely,\\
Pavel Shpagin\\
Vasyl Tereschenko

\end{document}

